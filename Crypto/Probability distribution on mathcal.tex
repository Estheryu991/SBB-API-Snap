By assumption we have for all $c ∈ \mathcal{C}$ that
$$Pr[\mathbf{c} = c] = \sum_{k \text{ with } c ∈ C(k)}
Pr[\mathbf[k] = k] Pr[\mathbf{p} = D_k{c}]
$$
where $C(k) := \lbrace{E_k(p) : p ∈ \mathcal{P}}\rbrace$. This yields
$$Pr[\mathbf{c} = c \mid \mathbf{p} = p] =
\sum_{k \text{ with } c = D_k(p)} Pr[\mathbf{k} = k]
$$
Bayes' theorem implies (if $Pr[\mathbf{c} = c] > 0$) that
$$Pr[\mathbf{p} = p \mid \mathbf{c} = c] =
\frac{Pr[\mathbf{p} = p] \; Pr[\mathbf{c} = c \mid \mathbf{p} = p]}
     {Pr[\mathbf{c} = c]}.
$$

Definition: A cryptosystem has perfect secrecy if $Pr[\mathbf{p} = p \mid \mathbf{c} = c] = Pr[\mathbf{p} = p]$ for all $p ∈ \mathcal{P},c ∈ \mathcal{C}$.
