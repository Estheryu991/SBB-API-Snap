$k$-tape Turing machine is a triple $M = (A,Q,τ)$ satisfying:

*  $A$ is a finite **alphabet** that the $k$ tapes contain:
   $A = \lbrace{\square, \vartriangleright, 0, 1}\rbrace$;
*  $Q$ is a finite set of **states** that includes $q_{start}, q_{halt} ∈ Q$;
*  $τ: Q × A^k → Q × A^{k−1} × \lbrace{L,S,R}\rbrace^k$ is the
   **transition function** of $M$.
   
First tape is input tape (read-only), second to $k-1$-st tape are work tapes, $k$-th tape is output tape.

$\square$ is the blank symbol, $\vartriangleright$ is the start symbol

$q_{start}, q_{halt}$ are the start and halting states


### Probabilistic Turing machine

A Turing machine that may choose at every step a move at random according to a
probability distribution.

Note that such a Turing machine is thus **non-deterministic**.

A probabilistic Turing machine $M$ computes a function
$f: \lbrace{0, 1}\rbrace^* → \lbrace{0, 1}\rbrace^*$ if
$$P[f(x) = M(x)] ≥ ⅔
$$

## Running time

$M$ computes $f: \lbrace{0, 1}\rbrace^* → \lbrace{0, 1}\rbrace^*$ in
$T(n)$-time, if for every input $x ∈ \lbrace{0, 1}\rbrace^*$, $M$ halts after at
most $T(|n|)$ steps with output $f(x)$.

### Running times of arithmetic operations

| Operation                                | Complexity |
| ---------------------------------------- | ---------- |
| Addition of two $n$-bit numbers          | $O(n)$     |
| Multiplication of two $n$-bit numbers    | $O(n^2)$   |
| Raising a number to an $n$-bit power     | $O(n^3)$   |
| Exhaustive key search for an $n$-bit key | $O(2^n)$   |
